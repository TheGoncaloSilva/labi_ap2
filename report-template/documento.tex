\documentclass{report}
\usepackage[T1]{fontenc} % Fontes T1
\usepackage[utf8]{inputenc} % Input UTF8
\usepackage[backend=biber, style=ieee]{biblatex} % para usar bibliografia
\usepackage{csquotes}
\usepackage[portuguese]{babel} %Usar língua portuguesa
\usepackage{blindtext} % Gerar texto automaticamente
\usepackage[printonlyused]{acronym}
\usepackage{hyperref} % para autoref
\usepackage{graphicx}

\bibliography{bibliografia}


\begin{document}
%%
% Definições
%
\def\titulo{TÍTULO DO RELATÓRIO}
\def\data{DATA}
\def\autores{Autor1, Autor2}
\def\autorescontactos{(nmec1) autor1@ua.pt, (nmec2) autor2@ua.pt}
\def\versao{VERSAO}
\def\departamento{DEPARTAMENTO}
\def\empresa{EMPRESA}
\def\logotipo{ua.pdf}
%
%%%%%% CAPA %%%%%%
%
\renewcommand{\contentsname}{Índice}
\begin{titlepage}

\begin{center}
%
\vspace*{50mm}
%
{\Huge \titulo}\\ 
%
\vspace{10mm}
%
{\Large \empresa}\\
%
\vspace{10mm}
%
{\LARGE \autores}\\ 
%
\vspace{30mm}
%
\begin{figure}[h]
\center
\includegraphics{\logotipo}
\end{figure}
%
\vspace{30mm}
\end{center}
%
\begin{flushright}
\versao
\end{flushright}
\end{titlepage}

%%  Página de Título %%
\title{%
{\Huge\textbf{\titulo}}\\
{\Large \departamento\\ \empresa}
}
%
\author{%
    \autores \\
    \autorescontactos
}
%
\date{\data}
%
\maketitle

\pagenumbering{roman}

%%%%%% RESUMO %%%%%%
\begin{abstract}
Este projeto foi realizado no âmbito da cadeira LABI (acr) do 1º ano do MIECT.
Consiste em criar um servidor em que os clientes se conectam para jogar um jogo de
adivinha o número secreto. Para além disso, também tivemos que fazer este mesmo 
relatório em que nós explicamos o projeto: objetivo, motivação, a metodologia utilizada, resultados e
conclusões. Na metodologia, será relatado em pormenor o nosso código feito para criar o servidor, tanto
código do cliente como do servidor, o modo de funcionamento, testagem e comandos git feitos para tal.
Nos resultados, será mostrado o fruto de todo o nosso código que é o servidor a funcionar.
Por fim, nas conclusões, retira-se o que se alcançou com este projeto, o que aprendemos,
o quão útil este projeto é para compreendermos esta matéria da cadeira de LABI e o 
quão interessante foi realizá-lo.
\end{abstract}

%%%%%% Agradecimentos %%%%%%
% Segundo glisc deveria aparecer após conclusão...
\renewcommand{\abstractname}{Agradecimentos}
\begin{abstract}
Eventuais agradecimentos.
Comentar bloco caso não existam agradecimentos a fazer.
\end{abstract}


\tableofcontents
% \listoftables     % descomentar se necessário
% \listoffigures    % descomentar se necessário


%%%%%%%%%%%%%%%%%%%%%%%%%%%%%%%
\clearpage
\pagenumbering{arabic}

%%%%%%%%%%%%%%%%%%%%%%%%%%%%%%%%
\chapter{Introdução}
\label{chap.introducao}

O objetivo do jogo é que o cliente adivinhe um número secreto entre 0 e 100 criado 
aleatoriamente pelo servidor, dentro de um determinado número de tentativas dadas também
pelo servidor. O cliente tem a possibilidade de realizar quatro operações diferentes. A primeira
serve para iniciar o jogo (START), a segunda serve para tentar adivinhar o número secreto (GUESS), 
a terceira serve para desistir do jogo (STOP) e a última serve para sair (QUIT). O servidor deve responder
adequadamente aquando da operação pretendida. Se houver algum erro, como um cliente inexistente, o 
servidor deve dar uma resposta adequada ao cliente. No final do jogo, o servidor guarda num ficheiro
.csv os dados do cliente que jogou o jogo. Para efeitos de segurança, o cliente possui ainda a
possibilidade de escolher se quer que os seus dados sejam encriptados ou não.

Este documento está dividido em quatro capítulos.
Depois desta introdução,
no \autoref{chap.metodologia} é apresentada a metodologia seguida,
no \autoref{chap.resultados} são apresentados os resultados obtidos,
sendo estes discutidos no \autoref{chap.analise}.
Finalmente, no \autoref{chap.conclusao} são apresentadas
as conclusões do trabalho.


\chapter{Metodologia}
\label{chap.metodologia}
\section{Cliente}
Nesta secção será apresentada a metodologica no ficheiro client.py

\subsection{Função main e run client}


\section{Servidor}


\section{Exemplos}


\chapter{Resultados}
\label{chap.resultados}
Descreve os resultados obtidos.

\chapter{Análise}
\label{chap.analise}
Analisa os resultados.

\chapter{Conclusões}
\label{chap.conclusao}
Apresenta conclusões.

\chapter*{Contribuições dos autores}
Resumir aqui o que cada autor fez no trabalho.
Usar abreviaturas para identificar os autores,
por exemplo AS para António Silva.
No fim indicar a percentagem de contribuição de cada autor.

%%%%%%%%%%%%%%%%%%%%%%%%%%%%%%%%%
\chapter*{Acrónimos}
\begin{acronym}
\acro{ua}[UA]{Universidade de Aveiro}
\acro{miect}[MIECT]{Mestrado Integrado em Engenharia de Computadores e Telemática}
\acro{lei}[LEI]{Licenciatura em Engenharia Informática}
\acro{glisc}[GLISC]{Grey Literature International Steering Committee}
\end{acronym}


%%%%%%%%%%%%%%%%%%%%%%%%%%%%%%%%%
\printbibliography

\end{document}
